\section{Introduction}
\frame{\sectionpage}

\begin{frame}{RD Identification}
    \begin{table}[h!]
    \begin{center}
        \begin{tabular}{ccccccc}
        
        & & \uncover<1->{$\underbrace{Z_i}_\text{running variable}$} & \uncover<3->{$\xRightarrow{W_i=\mathbf{1}\left(\{Z_i\geq {\textcolor<4->{mygreen}{c}}\}\right)}$} & \uncover<2->{$\underbrace{W_i}_\text{treatment}$} & \uncover<5->{$\Rightarrow$} & \uncover<5->{$\underbrace{Y_i}_{\text{outcome}}$}\\
        \uncover<6->{&\\
           \hline
           & \\
            & & test scores & & admission & & outcomes\\}
        \uncover<7->{& & test results && medication && outcomes}
        \end{tabular}
    \end{center}
    \end{table}

\end{frame}

\begin{frame}{RD Identification: Continuity Argument}
    For potential outcomes $\left\{ Y_i(0),Y_i(1) \right\}$: $Y_i=Y_i(W_i)$, a weighted \textcolor{mygreen}{\textbf{causal effect}} can be identified as 
    \begin{align*}
        \tau_c &= \mathbb{E}\left[ Y_i(1)-Y_i(0)\mid Z_i=c \right]\\
        \uncover<2->{&= \textcolor<3->{mygreen}{\lim_{z\downarrow c}}\mathbb{E}\left[Y\mid Z=z\right] - \textcolor<3->{mygreen}{\lim_{z\uparrow c}} \mathbb{E}\left[ Y\mid Z=z \right] }
    \end{align*}
        
    \uncover<3->{
        assuming
        \begin{itemize}
            \item the conditional response functions $\mu_{w}(z)=\mathbb{E}\left[Y(w)\mid Z=z\right]$ are \textcolor{mygreen}{\textbf{\underline{continuous}}}
            \item<4-> $\mu_w(z)$ to have a uniformly \textcolor{mygreen}{\textbf{\underline{bounded 2nd derivative}}} for CIs {\footnotesize\citep{armstrong2018optimal,armstrong2020simple}}
        \end{itemize}
    }

\end{frame}

\begin{frame}{RD Identification: Problems of Continuity Argument}
    Assumption: \textcolor{mygreen}{\textbf{\underline{continuous}}} $\mu_{w}(z)=\mathbb{E}\left[Y(w)\mid Z=z\right]$
    \begin{align*}
        \tau_c &= \lim_{z\downarrow c}\mathbb{E}\left[Y\mid Z=z\right] - \lim_{z\uparrow c} \mathbb{E}\left[ Y\mid Z=z \right]
    \end{align*}
    \uncover<1->{
        Where does this continuity come from?
    }

    \vspace*{15pt}
    \uncover<2->{
        \citet{lee2008randomized}: \textcolor{mygreen}{\textbf{\underline{continuous measurement error}}} in the running variable by units
    }

\end{frame}

\begin{frame}{RD Identification: Measurement Error}
    \begin{table}[h!]
    \begin{center}
        \begin{tabular}{ccccccc}
        
       \uncover<1->{$\textcolor<2->{mygreen}{\underbrace{U_i}_{\text{{latent variable}}}}$} & \uncover<2->{$\xRightarrow{Z_i\mid U_i\sim p(\cdot\mid U_i)}$} & $\underbrace{Z_i}_\text{running variable}$ & $\xRightarrow{W_i=\mathbf{1}\left(\{Z_i\geq {\textcolor{mygreen}{c}}\}\right)}$ & $\underbrace{W_i}_\text{treatment}$ & $\Rightarrow$ & $\underbrace{Y_i}_{\text{outcome}}$\\
        &\\
           \hline
           & \\
          \uncover<2->{\textcolor{mygreen}{ability}}  & & test scores & & admission & & outcomes\\
          \uncover<2->{\textcolor{mygreen}{condition}}  & & test results && medication && outcomes
        \end{tabular}
    \end{center}
    \end{table}

    \uncover<3->{
        \vspace*{15pt}
        Why don't we take advantage of the {\textcolor{mygreen}{\textbf{\underline{measurement error}}}} itself for inference?
    }

\end{frame}

\begin{frame}{This Paper}
    \begin{table}[h!]
        \begin{center}
            \begin{tabular}{ccccccc}
          $\underbrace{U_i}_{\text{{latent variable}}}$ & $\xRightarrow{ Z_i\mid U_i\sim \textcolor{mygreen}{p(\cdot\mid U_i)}}$ & $\underbrace{Z_i}_\text{running variable}$ & $\xRightarrow{W_i=\mathbf{1}\left(\{Z_i\geq {\textcolor{mygreen}{c}}\}\right)}$ & $\underbrace{W_i}_\text{treatment}$ & $\Rightarrow$ & $\underbrace{Y_i}_{\text{outcome}}$
            \end{tabular}
        \end{center}
        \end{table}

    Weighted treatment effects can be estimated if the measurement error in $Z_i$
    \begin{itemize}
        \item<2-> has a \textcolor{mygreen}{\textbf{\underline{known distribution}}}
        \item<3-> is \textcolor{mygreen}{\textbf{\underline{conditionally {\footnotesize(on $U_i$)} independent}}} of potential outcomes
    \end{itemize}
    
\end{frame}