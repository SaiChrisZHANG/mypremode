\usepackage{amsthm}
\usepackage{mathtools}
\usepackage{physics}
\usepackage{calligra}
\usepackage{csquotes}
\usepackage{tensor}
\usepackage[thicklines]{cancel}
\usepackage{tcolorbox}
\usepackage{pstricks}

\usepackage{multirow}
\usepackage{multicol}
\usepackage{bigdelim}

\usepackage{tabularx}
\usepackage{tikz}
\usepackage{mathtools}
\usepackage{amsmath,amssymb}

%\usepackage[font=footnotesize,labelfont={color=orange,bf}]{caption}
\usepackage{graphicx}

\usepackage[absolute,overlay]{textpos}
%\usepackage[texcoord,grid,gridcolor=red!10,subgridcolor=green!10,gridunit=pt]{eso-pic}

\usepackage{xparse}
\NewDocumentCommand{\framecolorbox}{oommm}
    {% #1 = width (optional)
    % #2 = inner alignment (optional)
    % #3 = frame color
    % #4 = background color
    % #5 = text
    \IfValueTF{#1}
    {%
    \IfValueTF{#2}
     {\fcolorbox{#3}{#4}{\makebox[#1][#2]{#5}}}
     {\fcolorbox{#3}{#4}{\makebox[#1]{#5}}}%
    }
    {\fcolorbox{#3}{#4}{#5}}%
    }

\usepackage[natbib=true,backend=biber,style=apa, sorting=nty, citestyle=authoryear-comp]{biblatex} %Custom bibliography
    \addbibresource{bib.bib} %Load references

\AtBeginBibliography{\footnotesize}

\DeclareMathAlphabet{\mathcalligra}{T1}{calligra}{m}{n}
\DeclareFontShape{T1}{calligra}{m}{n}{<->s*[2.2]callig15}{}
\newcommand{\scriptr}{\mathcalligra{r}\,}
\newcommand{\boldscriptr}{\pmb{\mathcalligra{r}}\,}
\def\rc{\scriptr}
\def\brc{\boldscriptr}
\def\hrc{\hat\brc}
\newcommand{\ie}{\emph{i.e.}} %id est
\newcommand{\eg}{\emph{e.g.}} %exempli gratia
\newcommand{\rtd}[1]{\ensuremath{\left\lfloor #1 \right\rfloor}}
\newcommand{\dirac}[1]{\ensuremath{\delta \left( #1 \right)}}
\newcommand{\diract}[1]{\ensuremath{\delta^3 \left( #1 \right)}}
\newcommand{\e}{\ensuremath{\epsilon_0}}
\newcommand{\m}{\ensuremath{\mu_0}}
\newcommand{\V}{\ensuremath{\mathcal{V}}}
\newcommand{\prnt}[1]{\ensuremath{\left(#1\right)}} %parentheses
\newcommand{\colch}[1]{\ensuremath{\left[#1\right]}} %square brackets
\newcommand{\chave}[1]{\ensuremath{\left\{#1\right\}}}  %curly brackets

\useoutertheme{infolines}
\useinnertheme{rectangles}
\usefonttheme{professionalfonts}


\definecolor{orange}{HTML}{f28165}
\definecolor{gray}{HTML}{303030}
\definecolor{yellow}{HTML}{f0be52}
\definecolor{lightorange}{HTML}{f19e58}
\definecolor{myblue}{cmyk}{1,.72,0,.38}
\definecolor{aliceblue}{rgb}{0.94, 0.97, 1.0}

\renewcommand{\CancelColor}{\color{orange}}

\makeatletter
\newcommand{\mybox}[1]{%
  \setbox0=\hbox{#1}%
  \setlength{\@tempdima}{\dimexpr\wd0+13pt}%
  \begin{tcolorbox}[colback=orange,colframe=orange,boxrule=0.5pt,arc=4pt,
      left=6pt,right=6pt,top=6pt,bottom=6pt,boxsep=0pt,width=\@tempdima]
    \textcolor{white}{#1}
  \end{tcolorbox}
}
\makeatother

\usecolortheme[named=orange]{structure}
\usecolortheme{sidebartab}
\usecolortheme{orchid}
\usecolortheme{whale}
\setbeamercolor{alerted text}{fg=yellow}
\setbeamercolor{block title alerted}{bg=alerted text.fg!90!black}
\setbeamercolor{block title example}{bg=lightorange!60!black}
\setbeamercolor{background canvas}{bg=gray}
\setbeamercolor{normal text}{bg=gray,fg=white}
\setbeamercolor{subsection in head/foot}{bg=white, fg=gray}


\setbeamertemplate{blocks}[rectangle]
\setbeamercovered{dynamic}

\setbeamertemplate{caption}[numbered]

\setbeamertemplate{section page}
{
	\begin{centering}
		\begin{beamercolorbox}[sep=27pt,center]{part title}
			\usebeamerfont{section title}\insertsection\par
			\usebeamerfont{subsection title}\insertsubsection\par
		\end{beamercolorbox}
	\end{centering}
}

\addtobeamertemplate{navigation symbols}{}{ \hspace{1em}    \usebeamerfont{footline}%
    \insertframenumber / \inserttotalframenumber }
\setbeamertemplate{navigation symbols}{}

%\beamer@compresstrue
\defbeamertemplate*{headline}{smoothbars theme}{%
  \begin{beamercolorbox}[ht=2.125ex,dp=3.150ex]{section in head/foot}
  \insertnavigation{\paperwidth}
  \end{beamercolorbox}%

  \begin{beamercolorbox}[ht=2.125ex,dp=1.125ex,%
  leftskip=.3cm,rightskip=.3cm plus1fil]{subsection in head/foot}
  \usebeamerfont{subsection in head/foot}\insertsubsectionhead
  \end{beamercolorbox}%
}

\setbeamertemplate{subsection page}
{
	\begin{centering}
		\begin{beamercolorbox}[sep=12pt,center]{part title}
			\usebeamerfont{subsection title}\insertsubsection\par
		\end{beamercolorbox}
	\end{centering}
}

\newcommand{\hlight}[1]{\colorbox{violet!50}{#1}}
\newcommand{\hlighta}[1]{\colorbox{red!50}{#1}}

\newcommand{\boxorange}[1]{
\begin{center}
\fcolorbox{orange}{gray}{
\begin{minipage}{0.95\textwidth}
#1
\end{minipage}
}
\end{center}
}

\newcommand{\boxgrey}[1]{
\begin{center}
\fcolorbox{black!85!white}{lightgray!35!white}{
\begin{minipage}{0.8\textwidth}
#1
\end{minipage}
}
\end{center}
}

\setlength{\abovecaptionskip}{5pt plus 3pt minus 2pt}

% Block colors
\colorlet{orangeTitleBlockColor}{orange}
\colorlet{orangeBlockColor}{orange!25!gray}
\colorlet{blockTitleTextColor}{gray}
\colorlet{blockBodyTextColor}{white}

\setbeamertemplate{blocks}[rectangle]

\setbeamercolor*{block title}{
  fg=blockTitleTextColor,
  bg=orangeTitleBlockColor}
\setbeamercolor*{block body}{
  fg=blockBodyTextColor,
  bg=orangeBlockColor}
  
\setbeamerfont{block title}{size={}}

\AtBeginEnvironment{block}{%
  \setbeamercolor{itemize item}{fg=orangeTitleBlockColor!70}}
  