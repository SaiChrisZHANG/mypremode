\section{Introduction}
\frame{\sectionpage}

\begin{frame}{Automatic Prejudice}
    \uncover<+->{Racial prejudice and stereotyping can operate \textcolor{lightlavender!55!white}{\textbf{automatically}}
    
    \hfill {\scriptsize \citet{bargh1994four,bargh1999cognitive}}}
    
    \uncover<+->{and they may be controlled \textcolor{lightlavender!55!white}{\textbf{consciously}}
    
    \hfill {\scriptsize \citet{bargh1999cognitive,devine1999automaticity}}}
    
    \vspace*{10pt}
    \uncover<+->{ \textcolor{lightlavender!55!white}{\textbf{Dual-processing}} approaches to social cognition {\scriptsize \citet{chaiken1999dual}}
    \vspace*{-5pt}
        \begin{columns}[T]
            \begin{column}{0.45\textwidth}
                \begin{block}{\textbf{Activation} {\scriptsize \citet{devine1989stereotypes}}}
                    upon \textbf{exposure}, irrespective of \textit{conscious intentions, beliefs or prejudice}
                \end{block}
            \end{column}

            \begin{column}{0.45\textwidth}
                \begin{block}{\textbf{Application} {\scriptsize \citet{plant1998internal}}}
                    susceptible to \textit{conscious control}
                \end{block}
            \end{column}
        \end{columns}
    }
\end{frame}

\begin{frame}{\textit{Control} Prejudice}
    \begin{columns}[T]
        \begin{column}{0.8\textwidth}
            \vspace*{-2pt}
            \uncover<+->{
                \begin{block}{controllability of \textbf{application}}
                    4 conditions {\footnotesize \citep{bargh1994four,bargh1999cognitive}}:
                    \begin{itemize}
                        \small
                        \item \underline{\textit{awareness}} of invoking a stereotype 
                        \item \underline{\textit{understanding}} of the stereotype affects judgment 
                        \item \underline{\textit{motivation}} to control the stereotype
                        \item \underline{\textit{capacity}} to act on the motivation
                    \end{itemize}
                \end{block}
            }

            \uncover<+->{\begin{block}{automaticity of \textbf{activation}}
                might \textbf{NOT} be inevitable {\footnotesize \citep{bargh1999cognitive,blair2001imagining,dasgupta2000automatic,moskowitz2000preconsciously}}
               \end{block}
               \vspace*{10pt}}
        \end{column}
    \end{columns}
    
\end{frame}

\begin{frame}{Control Stereotype \textit{Activation}}
    Automatic stereotype activation is the product of \textcolor{lightlavender!55!white}{\underline{\textbf{long-term exposure}}} to particular associations:
    \begin{itemize}
        \item<+-> \underline{\textit{\citet{kawakami2000just}}}: experimentally generated counterstereotypical associations can \textcolor{lightlavender!55!white}{\textbf{temporarily}} eliminate stereotyped associations
        \item<+-> \underline{\textit{\citet{bargh1999cognitive,moskowitz1999preconscious}}}: hypothetically, people with nonprejudiced motivations may over time \textcolor{lightlavender!55!white}{\textbf{learn}} to associate ealitarian thoughts with members of stereotyped groups, until the new association becomes automatized
    \end{itemize}
\end{frame}

\begin{frame}{This Study: A \textit{Contemporary Communication Theory} Approach}
    \textcolor{lightlavender!55!white}{\underline{\textit{Hypothesis}}}: social interaction is predicated on \underline{\textbf{ongoing mutual perspective taking}} that is \textcolor{lightlavender!55!white}{highly routinized}, and \textcolor{lightlavender!55!white}{probably automatic} {\footnotesize \citep{clark1996using}}
    $$
        \text{\large {\small establishment and maintenance} of \textcolor{lightlavender!55!white}{Common Ground}}
    $$
    \vspace*{-15pt}
    \uncover<2->{\begin{itemize}
        \item adjust perspectives and communicative attempts according to inferences they make about the \textcolor{lightlavender!55!white}{\textbf{knowledge}} and \textcolor{lightlavender!55!white}{\textbf{attitudes}} of others
        \begin{itemize}
            \item<3-> influential incidental knowledge {\footnotesize \citep{mccann1992personal}}
            \item<4-> depends on relationship-relevant motives {\footnotesize \citep{mccann1983self,higgins1984social}}
        \end{itemize}
    \end{itemize}}
\end{frame}

\begin{frame}{This Study: Social Tunning and Prejudice}
    \begin{columns}[T]
        \begin{column}{0.45\textwidth}
            \begin{block}{\textbf{social desirability}}
                \begin{itemize}
                    \item it is an important concern {\footnotesize \citep{crosby1980recent}}
                    \item how to bypass it {\footnotesize \citep{greenwald1995implicit,fazio1995variability} }
                \end{itemize}
            \end{block}
        \end{column}

        \begin{column}{0.45\textwidth}
            \begin{block}{\textbf{attitude change}}
                \begin{itemize}
                    \item influence on private attitudes
                    \begin{itemize}
                        \footnotesize
                        \item \textit{\underline{assimilation}} \citep{blanchard1991reducing}
                        \item \textit{\underline{social motivation reducing} \underline{automatic activation}} \citep{sinclair1999reactions}
                    \end{itemize}
                \end{itemize}
            \end{block}
        \end{column}
    \end{columns}
\end{frame}

\begin{frame}{This Study: Road Map}
    \begin{itemize}
        \item<+-> Main goal: identify \textcolor{lightlavender!55!white}{\textit{\underline{conditions}}} in which social tuning might be observed on automatic racial prejudice measures, in the presence of \textcolor{lightlavender!55!white}{\textit{\underline{Black versus White}}} experimenters
        \item<+-> Variation:
        \begin{itemize}
            \item<+->[\textcolor{lightlavender!55!white}{\textbf{E1}}] presence of a \textcolor{lightlavender!55!white}{\textit{\underline{Black}}} versus \textcolor{lightlavender!55!white}{\textit{\underline{White}}} experimenter
            \begin{itemize}
                \item[-] \textit{\underline{automatic activation}}: more anti-Black prejudice in the presence of a Black experimenter
                \item[-] \textit{\underline{social tuning}}: less automatic racial prejudice may be observed
            \end{itemize}
            \item<+->[\textcolor{lightlavender!55!white}{\textbf{E2}}] subjects' race: \textcolor{lightlavender!55!white}{\textit{\underline{Asian American}}} versus \textcolor{lightlavender!55!white}{\textit{\underline{European American}}}
            \item<+->[\textcolor{lightlavender!55!white}{\textbf{E3}}] \textcolor{lightlavender!55!white}{\textit{\underline{tacit}}} versus \textcolor{lightlavender!55!white}{\textit{\underline{expressed}}} social influence 
            \item<+->[\textcolor{lightlavender!55!white}{\textbf{E4}}] prejudice measures: \textcolor{lightlavender!55!white}{\textit{\underline{Implicit Association Test}}} {\scriptsize\citep[IAT,][]{greenwald1998measuring}} versus \textcolor{lightlavender!55!white}{\textit{\underline{a subliminal priming measure}}}
        \end{itemize}
    \end{itemize}


\end{frame}
